\section{Διάρθρωση της Αναφοράς}
\label{section:layout}

Η διάρθρωση της παρούσας διπλωματικής εργασίας είναι η εξής:

\begin{itemize}
  \item {Στο \textbf{Μέρος Ι} περιγράφονται ορισμένες εισαγωγικές έννοιες και
      βασικές γνώσεις, ώστε ο αναγνώστης να κατανοήσει με μεγαλύτερη ευκολία
      τα αντικείμενα που πραγματεύεται η διπλωματική εργασία.
      Ειδικότερα, τα κεφάλαια είναι:
      \begin{itemize}
        \item{\textbf{Κεφάλαιο 2:} Παρατίθεται η ανασκόπηση της ερευνητικής
            περιοχής αναφορικά με τα αντικείμενα στα οποία επιδιώκει να
            παρουσιάσει λύσεις η διπλωματική εργασία.
          }
        \item{\textbf{Κεφάλαιο 3:} Περιγράφονται τα βασικά θεωρητικά στοιχεία
            στα οποία βασίστηκε η παρούσα υλοποίηση, καθώς επίσης και οι
            διάφορες τεχνικές και τα εργαλεία που χρησιμοποιήθηκαν.
          }
      \end{itemize}
    }
  \item{Το \textbf{Μέρος ΙΙ} αποτελείται από τρία κεφάλαια στα οποία
      περιγράφονται πλήρως η υλοποιήσεις.
      %\begin{itemize}
      %\end{itemize}
    }
  \item{Στο \textbf{Μέρος ΙΙΙ} παρουσιάζεται αναλυτικά η μεθοδολογία των
      πειραμάτων, τα αποτελέσματά τους και τα συμπεράσματα στα οποία καταλήξαμε.
      Τέλος, γίνεται αναφορά σε διαδικαστικά προβλήματα που ανέκυψαν
      και προτείνονται θέματα για μελλοντική μελέτη.
      %\begin{itemize}
      %\end{itemize}
    }
\end{itemize}



